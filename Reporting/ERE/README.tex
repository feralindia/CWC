% This file was converted from HTML to LaTeX with
% gnuhtml2latex program
% (c) Tomasz Wegrzanowski <maniek@beer.com> 1999
% (c) Gunnar Wolf <gwolf@gwolf.org> 2005-2010
% Version : 0.4.
\documentclass{article}
\usepackage[utf8]{inputenc}
\begin{document}


\section*{Extreme Rainfall Events in the Western Ghats}

\subsection*{Impacts of Land Cover on Discharge Volume and Quality}

\section*{Script Description}

\subsection*{Objectives}

\begin{enumerate}
\item Identify extreme rainfall events in statistical terms
\item Determine discharge from associated streams for each of these events
\item Identify dominant land cover for each of the stream catchments 
\item Determine nutrient content in the discharge
\item Investigate statistical relationships
\item Investigate ecological ramifications
\end{enumerate}

\section*{Identify extreme rainfall events in statistical terms}

\par Goswami et al., identify ERE at 150mm per day for central India on the basis that this area had fairly homogenous seasonal as well as daily variabiliy in rainfall. On the other hand they clarified that a fixed threshhold would not be appropirate for areas with high variability.

\par Hence we adopted a statistical approach based on our own datasets. We computed the outliers for each of the sites separately based on the 95th percentile based on a chi square score: \texttt{(x - mean(x))\^{}2/var(x)} as implemented by the outliers package and described here.

\section*{Determine discharge from associated streams for each of these events}

\section*{Identify dominant land cover for each of the stream catchments}

\section*{Determine nutrient content in the discharge}

\section*{Investigate statistical relationships}

\section*{Investigate ecological ramifications}

\end{document}

